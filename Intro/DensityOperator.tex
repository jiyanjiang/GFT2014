\section{系综}

\subsection{宏观和微观}

统计物理研究的对象是“大数自由度”的系统在宏观长时间上的平均行为。“宏观”指的是空间,空间的尺度较大,“大尺度”和“长时间”是针对特定物理系统,物理过程而言的。

对原子大小($\sim 10^{-10}m$)大小的物理系统而言,特征能量是$\sim eV$,特征时间按不确定关系估算:$\Delta \tau \Delta E \sim \hbar$,我们可计算出原子大小物理系统的特征时间:

\begin{equation*}
\Delta \tau \sim \frac{\hbar}{1eV} \sim 10^{-15}s,
\end{equation*}

对人(观察者)有意义的时间尺度是:$1s$(脉搏), 这里$10^{-15}s$就是微观的时间尺度,$1s$就是宏观的时间尺度,其比例是:$10^{15}$。

类似我们可以估计空间的比例,典型的微观,原子大小$\sim 10^{-10}m$,典型的宏观, $1m$, 比例是:$10^{10}$。

“大自由度”,比如考虑$22.4 l$的理想气体,对人(观察者)有意义的物理量仅有P(压强),V(体积),T(温度)几个,或者说自由度很小。但如果采用微观陈述,这个系统就包括$6.02 \times
10^{23}$个粒子,自由度的比例是:$10^{23}$。

涉及大数物理规律的偏差是$\frac{1}{\sqrt N}$,$N$很大保证了统计物理规律是非常准确的\footnote{参考:薛定谔,《生命是什么》,第一章}。



\subsection{密度算符}

\subsubsection{纯系综}

统计物理中,我们用系综理论建立“微观---宏观”'的联系。所谓系综就是想象中的集合,对量子力学而言,即便不涉及统计物理,我们也需要系综概念以建立测量理论。

假设量子态$\left| \alpha \right\rangle$,测量力学量A,如果$\left| \alpha \right\rangle$不是A的本征态,我们无法推测测量值是多少,我们只能说可能是多少,算式$\left\langle \alpha \right| A \left| \alpha \right\rangle$给出的是所谓期望值,我们可通过一个假想的操作来得到这个期望值。

\begin{enumerate}

\item 在想象中把态矢$\left| \alpha \right\rangle$复制N份;

\item 取出第一个$\left| \alpha \right\rangle$,测量A,得到$a_1$,但一旦完成测量,物理系统就被改变了,成为$\left| a_1 \right\rangle$,所以我们无法再继续测量了;

\item 取出第二个$\left| \alpha \right\rangle$测量A,得到$a_2$,依次得到$a_3, a_4, ...$

\item 当N足够大时,$\frac{\sum\limits_{i} a_i}{N}$收敛,这个值就是A的期望值。

\end{enumerate}

这样操作,得到的系综叫纯系综,因为它就对应某个确定的态矢量$\left| \alpha \right\rangle$。

\subsubsection{混合系综}

统计物理需要混合系综,即我们要研究的物理系统处在某个热力学分布中,$w_1$比例处于$\left| \alpha_1 \right\rangle$, $w_2$比例处于$\left| \alpha_2 \right\rangle$,...,并且满足:

\begin{equation}
\sum_n w_n =1.
\end{equation}

对混合系综,力学量$A$的平均值是:

\begin{equation}\label{statistic average}
[A] = \sum\limits_n w_n \left\langle a_n
\right| A \left| a_n \right\rangle.
\end{equation}

这里涉及两次平均,$\left\langle \alpha_n \right| A \left| \alpha_n \right\rangle$是量子力学平均,$\sum\limits_n ... $与系综分布有关,是热力学平均。

如何表达混合系综是个问题,比如我们无法把它表示为传统的波函数按线性迭加因子相加的形式:

\begin{equation*}
\psi = c_1 \psi_1 + c_2 \psi_2
\end{equation*}

因为这样写,暗含着$\psi_1$和$\psi_2$之间的相位差就固定了,或者说$\psi_1$和$\psi_2$是相干的。

我们把上式改写为:

\begin{equation*}
\psi = c_1 \psi_1 + c_2 e^{i \theta} \psi_2
\end{equation*}

这里引入了$\psi_2$和$\psi_1$之间的相位差$\theta$,如果说$\psi_1$, $\psi_2$完全不相干,就意味着$\theta$是个随机数,在$[0, 2\pi]$区间里等几率地随机取值。为了书写的方便,以下我们假设$c_1$, $c_2$都是实数。

计算力学量$A$的期望值:

\begin{equation*}
[A] = c_1^2 A_1 + c_2^2 A_2 + c_1 c_2 e^{i \theta} \left\langle \psi_1 | A | \psi_2 \right\rangle + c.c. 
\end{equation*}

这里$c.c.$表示复共轭,$e^{i \theta}$平均而言为0,因此:

\begin{equation*}
[A] = c_1^2 A_1 + c_2^2 A_2
\end{equation*}

我们令$c_1^2 = w_1$, $c_2^2 = w_2$,就得到混合系综平均的定义式(\ref{statistic average})。

密度算符(Density operator)是朗道引入以描述统计物理的,

\begin{equation}\label{statistic operator}
\rho = \sum\limits_n \left| a_n \right\rangle w_n \left\langle a_n \right|
\end{equation}

平均值(\ref{statistic average})可改写为:

\begin{equation}
[A] = Tr (\rho A)
\end{equation}

对统计算符$\rho$求偏导$i\hbar \frac{\partial}{\partial t}$,
可推出$\rho$满足的运动方程,“量子刘维方程”(Quantum Liouville Equation),


\begin{equation}\label{quantum Liouville eqs}
i\hbar \frac{\partial}{\partial t} \rho = H \rho - \rho H = [H,
\rho].
\end{equation}

注意:和海森堡运动方程比较$i\hbar \dot a = [a, H]$,
量子刘维方程在形式上差一个“负号”。

\subsection{正则分布}

对平衡态而言,$\rho$不随时间改变,因此$[H, \rho]=0$。这意味着如果$\rho$是$H$的幂函数,则$\rho$表示的态一定是平衡态。

正则分布,

\begin{equation}\label{canonical distribution}
\rho = \frac{e^{-\beta H}}{Z} = Z^{-1} e^{-\beta H}
\end{equation}

这里,$\beta = \frac{1}{k_B T}$,$Z^{-1}$是归一化因子,

\begin{equation*}
Z=Tr e^{-\beta H}
\end{equation*}

$Z$是个数,

\begin{equation*}
Z = Tr e^{-\beta H} = \sum_i \left\langle i \right| e^{-\beta H}
\left| i \right\rangle = \sum\limits_i e^{-\beta E_i}
\end{equation*}

正则系综的能量$E$,

\begin{equation*}
E = Tr (H \rho) = Z^{-1} Tr (H e^{-\beta H}),
\end{equation*}

因为$Z^{-1}$是个数,所以可提出$Tr$之前。

现在考虑正则系综的熵(entropy),

熵是不确定(或信息缺乏)的度量,假设等几率分布,几率$p$,状态数$\Gamma = \frac{1}{p}$,熵被定义为: $S = k_B \ln \Gamma = - k_B \ln p$。把这个定义推广为非等几率${p_{\lambda}}, \sum_{\lambda}
p_{\lambda}=1$分布,$S = -k_B \sum\limits_{\lambda} p_{\lambda}\ln
p_{\lambda}$。如果用密度算符表示的话,就是:

\begin{equation}\label{definition of entropy}
S = - k_B Tr (\rho \ln \rho)
\end{equation}

这里,

\begin{equation*}
\ln \rho = \ln ( Z^{-1} e^{-\beta H} ) = \ln e^{-\beta H} - \ln Z =
-\beta H - \ln Z,
\end{equation*}

那么,

\begin{equation*}
S = - k_B Tr (\rho \ln \rho) = k_B Tr [ \rho (\beta H + \ln Z) ].
\end{equation*}

第一项: $k_B \beta Tr (\rho H) = \frac{1}{T} E$,

第二项: $k_B Tr [ \rho \ln Z  ]$, 考虑到$Tr (\rho) = 1$, 这一项是: $k_B \ln Z$.

因此,

\begin{equation*}
S = \frac{E}{T} + k_B \ln Z
\end{equation*}

定义自由能(Free energy),

\begin{equation}\label{definition of free energy}
F = - k_B T \ln Z = - \beta^{-1} \ln Z,
\end{equation}

对正则系综,$E$,$S$,$F$存在关系:

\begin{equation}\label{relation ESF}
F = E - TS
\end{equation}

正则系综代表与热库接触,允许能量交换,保持温度固定而达到的平衡态。

假如不仅允许能量交换,还允许系统和热库交换粒子,保持温度$T$和化学势$\mu$固定,这样达到的平衡态,就是巨正则分布(Grand
canonical distribution)了,密度算符是:

\begin{equation}\label{density operator: GCE}
\rho_G = Z_G^{-1} e^{-\beta (H - \mu N)}
\end{equation}

这里$H, N$是算符, $Z_G^{-1}$是巨配分函数:

\begin{equation*}
Z_G = Tr e^{-\beta (H - \mu N)}
\end{equation*}

利用: $E = Tr (\rho H)$, $N = Tr (\rho N)$, 可证明:

\begin{equation*}
-k_B T \ln Z_G = \Omega = E- TS - \mu N
\end{equation*}

这里$\Omega(T,V,\mu)$是“广势函数”。由$\Omega$,我们可求出$S$, $P$, $N$:

\begin{eqnarray*}
% \nonumber to remove numbering (before each equation)
  S &=& -\left( \frac{\partial \Omega}{\partial T}\right)_{V, \mu} \\
  P &=& -\left( \frac{\partial \Omega}{\partial V} \right)_{T, \mu} \\
  N &=& -\left( \frac{\partial \Omega}{\partial \mu} \right)_{T, V}
\end{eqnarray*}

\subsection{热力学关系}

$T$, $P$, $\mu$这样的物理量叫强度量, $S$, $V$, $N$这样的量叫广延量。能量的变化$dE$可表示为,

\begin{equation}\label{dE relation}
dE = TdS - P dV + \mu dN,
\end{equation}

$-PdV$项中的负号意味着,体积增大,系统对外作功,系统内能E是减小的。公式(\ref{dE relation})表示,$E$是$S$, $V$, $N$的函数,即$E(S, V, N)$,由公式(\ref{dE relation}), 我们还可得到:

\begin{eqnarray*}
% \nonumber to remove numbering (before each equation)
  T &=& \left( \frac{\partial E}{\partial S} \right)_{V,N}  \\
  P &=& - \left(\frac{\partial E}{\partial V} \right)_{S,N} \\
  \mu &=& \left(\frac{\partial E}{\partial N} \right)_{S,V}
\end{eqnarray*}

现在我们作勒让德变换, 努力把$S$, $V$, $N$宗量变为, 比如$T$, $V$, $N$宗量.

\begin{equation*}
F = E - TS
\end{equation*}

则, $d F = d E - T dS - S dT = -S dT - P dV + \mu dN$, $F$是$T$, $V$, $N$的函数, 记为: $F(T, V, N)$, 求偏微分可得:

\begin{eqnarray*}
% \nonumber to remove numbering (before each equation)
  S &=& - \left( \frac{\partial F}{\partial T} \right)_{V,N} \\
  P &=& - \left( \frac{\partial F}{\partial V} \right)_{T,N} \\
  \mu &=& \left( \frac{\partial F}{\partial N} \right)_{T,V}
\end{eqnarray*}

继续作勒让德变换, 变成$T$, $V$, $\mu$宗量的,

\begin{equation*}
\Omega(T,V,\mu) = F - \mu N = E - TS - \mu N.
\end{equation*}

则, $d \Omega = -S dT - P dV - N d\mu$, 由此式求偏微分就得到$S$, $P$, $N$.

\subsection*{练习}

\begin{enumerate}

\item

证明:$[A] = Tr \rho A$

证:

\begin{equation}
Tr \rho A = \sum_i (\rho A)_{ii} = \sum_{ij} \rho_{ij} A_{ji} 
\end{equation}

这里:

\begin{equation}
A_{ji} = \left\langle j | A | i \right\rangle
\end{equation}

代入可得:

\begin{equation}
Tr \rho A = \sum_{ij} \sum_n   \left\langle i  | n \right\rangle w_n \left\langle n | j \right\rangle \left\langle j | A | i \right\rangle
\end{equation}

即:

\begin{equation}
Tr \rho A =  \sum_n   w_n \left\langle n | A | n \right\rangle
\end{equation}

QED

\item 

证明:$Tr \rho =1$

\item

推导密度算符随时间的演化,即量子刘维方程(Quantum Liouville equation),

\begin{equation}
i \hbar \frac{\partial }{\partial t } \rho = [H, \rho]
\end{equation}

提示:利用薛定谔方程和,$d (uv) = (du) v + u dv$

\end{enumerate}


\subsection*{阅读}

\begin{enumerate}
\item 

蔡建华, 龚昌德等, 《量子统计的格林函数理论》, $\S$ 1.1, $\S$ 1.2;

\item

J. J. Sakurai, Modern Quantum Mechanics, $\S$ 3.4;

\end{enumerate}




